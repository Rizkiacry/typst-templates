\documentclass[sigconf, language=english, language=german,language=french]{acmart}

\usepackage{layout}
\usepackage{showframe}

\begin{document}
\title{The Name of the Title Is Hope}

\author{Ben Trovato}
\authornote{Both authors contributed equally to this research.}
\email{trovato@corporation.com}
\orcid{1234-5678-9012}
\author{G.K.M. Tobin}
\authornotemark[1]
\email{webmaster@marysville-ohio.com}
\affiliation{%
  \institution{Institute for Clarity in Documentation}
  \city{Dublin}
  \state{Ohio}
  \country{USA}
}

\author{Lars Th{\o}rv{\"a}ld}
\affiliation{%
\institution{The Th{\o}rv{\"a}ld Group}
\city{Hekla}
\country{Iceland}}
\email{larst@affiliation.org}

\author{Valerie B\'eranger}
\affiliation{%
  \institution{Inria Paris-Rocquencourt}
  \city{Rocquencourt}
  \country{France}
}

\author{Aparna Patel}
\affiliation{%
  \institution{Rajiv Gandhi University}
  \city{Doimukh}
  \state{Arunachal Pradesh}
  \country{India}}

\author{Huifen Chan}
\affiliation{%
  \institution{Tsinghua University}
  \city{Haidian Qu}
  \state{Beijing Shi}
  \country{China}}

\author{Charles Palmer}
\affiliation{%
  \institution{Palmer Research Laboratories}
  \city{San Antonio}
  \state{Texas}
  \country{USA}}
\email{cpalmer@prl.com}

\author{John Smith}
\affiliation{%
\institution{The Th{\o}rv{\"a}ld Group}
\city{Hekla}
\country{Iceland}}
\email{jsmith@affiliation.org}

\author{Julius P. Kumquat}
\affiliation{%
  \institution{The Kumquat Consortium}
  \city{New York}
  \country{USA}}
\email{jpkumquat@consortium.net}

\begin{abstract}
  A clear and well-documented \LaTeX\ document is presented as an
  article formatted for publication by ACM in a conference proceedings
  or journal publication. Based on the ``acmart'' document class, this
  article presents and explains many of the common variations, as well
  as many of the formatting elements an author may use in the
  preparation of the documentation of their work.
\end{abstract}

\begin{translatedabstract}{french}
  Un document \LaTeX\ clair et bien documenté est présenté comme un
  article formaté pour publication par l'ACM dans un compte rendu de
  conférence ou une publication de revue. Basé sur la classe de
  document ``acmart'', cet article présente et explique de nombreuses
  des variations courantes, ainsi que de nombreux éléments de
  formatage qu'un auteur peut utiliser dans la préparation de la
  documentation de son travail.
\end{translatedabstract}

\begin{translatedabstract}{german}
  Un document \LaTeX\ clair et bien documenté est présenté comme un
  article formaté pour publication par l'ACM dans un compte rendu de
  conférence ou une publication de revue. Basé sur la classe de
  document ``acmart'', cet article présente et explique de nombreuses
  des variations courantes, ainsi que de nombreux éléments de
  formatage qu'un auteur peut utiliser dans la préparation de la
  documentation de son travail.
\end{translatedabstract}

\maketitle
\section{Introduction}
Text.

% \layout%

\makeatletter
base: \f@size

\verb+\tiny+ \tiny \f@size
\verb+\scriptsize+ \scriptsize \f@size
\verb+\footnotesize+ \footnotesize \f@size
\verb+\small+ \small \f@size
\verb+\normalsize+ \normalsize \f@size
\verb+\large+ \large \f@size
\verb+\Large+ \Large \f@size
\verb+\LARGE+ \LARGE \f@size
\verb+\huge+ \huge \f@size
\verb+\Huge+ \Huge \f@size
\makeatother

\end{document}