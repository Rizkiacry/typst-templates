\documentclass[
    a4paper,
    11pt,
    DIV=12,
    BCOR=10mm,          % Books usually have larger binding correction
    twoside,
    headings=normal,
    open=right,         % Chapters start on right-hand (recto) pages
    toc=bibliography,
    toc=listof,
    listof=totoc,
    parskip=false,      % Traditional book style: indentation, no parskip
    captions=tableheading,
    numbers=autoendperiod,
    appendixprefix=true,
    chapterprefix=true
]{scrbook}

\usepackage[utf8]{inputenc}
\usepackage[T1]{fontenc}
\usepackage{lmodern}
\usepackage[english]{babel}

\KOMAoptions{fontsize=10.5pt} % Example of a non-standard size
\KOMAoptions{DIV=10}

\usepackage{amsmath}
\usepackage{amsfonts}
\usepackage{amssymb}
\usepackage{scrlayer-scrpage}
\usepackage{listings}
\usepackage{longtable}
\usepackage{microtype}
\usepackage{xcolor}
\usepackage{lipsum}
\usepackage{makeidx}
\makeindex

% --- KOMA-Script Font Customizations ---
\setkomafont{disposition}{\normalfont\bfseries\Large}
\setkomafont{chapter}{\Huge\scshape\color{darkgray}}
\addtokomafont{section}{\large\color{blue!80!black}}
\setkomafont{captionlabel}{\bfseries}
\setkomafont{caption}{\itshape}
\setkomafont{pagenumber}{\small\sffamily}

% --- Header and Footer Configuration ---
\clearpairofpagestyles
\automark[section]{chapter}

% For scrheadings style (main matter)
\ihead{\leftmark}  % Inner: Chapter Number and Title
\ohead{\rightmark} % Outer: Section Number and Title
\ifoot{}
\ofoot{\pagemark}  % Outer: Page number

% For plain.scrheadings style (e.g., first page of chapter)
% Optional arguments apply to plain style
\ihead[]{\leftmark}
\ohead[]{\rightmark}
\ifoot[]{}
\ofoot[]{\pagemark}

\pagestyle{scrheadings}

% --- Listings Configuration ---
\lstset{
    language=Java,
    basicstyle=\ttfamily\scriptsize,
    keywordstyle=\color{red!70!black}\bfseries,
    commentstyle=\color{gray}\itshape,
    stringstyle=\color{orange!70!black},
    showstringspaces=false,
    breaklines=true,
    frame=lines,
    numbers=left,
    numberstyle=\tiny\color{black!50},
    captionpos=b
}

% --- Title Information ---
\title{The Art of KOMA-Script Book Design}
\subtitle{A Deep Dive into \texttt{scrbook}}
\author{Johannes Gutenberg Jr.}
\publishers{KOMA Publishing House}
\date{MMXXIV}
\titlehead{\centering \textit{A Work in Progress}}
\subject{Advanced LaTeX Typesetting}

% \dedicationpage is specific to scrbook to create a separate dedication page
\dedication{To the pursuit of beautiful documents.}

\begin{document}

\frontmatter % Starts roman page numbering, no chapter numbers in ToC/Headings

\maketitle
% Dedication page is output by \maketitle if \dedication was used

% \begin{abstract} % Abstract is usually part of front matter in books
%     \lipsum[1]
%     This book explores the capabilities of the \texttt{scrbook} class, demonstrating its power for professional book production.
% \end{abstract}
\clearpage

\tableofcontents
\listoftables
\lstlistoflistings%
\clearpage

\mainmatter % Starts arabic page numbering, resets page counter, enables chapter numbers

\part{KOMA-Script Essentials}
\chapter{The Philosophy of KOMA-Script}
\lipsum[2-4]
A footnote\footnote{Footnotes in books often have more detailed content.} and a \marginpar{Side note!} reference to the KOMA-Script logo: \KOMAScript.

\section{Principles}
\lipsum[5-6]
\begin{enumerate}
  \item Configurability
  \item Typographical Excellence
  \item Adherence to Standards (with flexibility)
\end{enumerate}

\section{Comparison with Standard Classes}
\lipsum[7-8]
This is \textit{emphasized} and this is \textbf{strong}.

\chapter{Advanced Layout Techniques}
\section{Utilizing Layers and Page Styles}
\lipsum[9-10]
KOMA-Script's \verb|scrlayer-scrpage| provides powerful tools.

\section{Working with Tables}
\begin{table}[hbtp]
  \caption{Comparative Analysis of Features}
  \label{tab:comparison}
  \centering
  \begin{tabular}{lcc}
    \hline
    Feature              & Standard LaTeX & KOMA-Script                   \\
    \hline
    Typearea Calculation & Manual/Basic   & Advanced (DIV, BCOR)          \\
    Heading Styles       & Limited        & Highly Customizable           \\
    Caption Placement    & Basic          & Flexible (e.g., tableheading) \\
    Font Elements        & Basic          & Granular Control              \\
    \hline
  \end{tabular}
\end{table}
\lipsum[11]

And a long table example for the book:
\begin{longtable}{lp{0.65\textwidth}}
  \caption{Detailed KOMA-Script Options Overview} \label{tab:komaoptions_long} \\
  \hline
  Option     & Description and Notes                                           \\
  \hline
  \endfirsthead
  \multicolumn{2}{l}%
  {{\bfseries Table \thetable\ continued from previous page}}                  \\
  \hline
  Option     & Description and Notes                                           \\
  \hline
  \endhead
  \hline \multicolumn{2}{r}{{Continued on next page}}                          \\
  \endfoot
  \hline
  \endlastfoot
  `DIV`      & \lipsum[1][1-3]                                                 \\
  `BCOR`     & \lipsum[2][1-3]                                                 \\
  `headings` & \lipsum[3][1-3]                                                 \\
  `parskip`  & \lipsum[4][1-3]                                                 \\
  `captions` & \lipsum[5][1-3]                                                 \\
\end{longtable}

\section{Code Examples}
\begin{lstlisting}[caption={A Java Snippet}, label={lst:java}]
public class HelloWorld {
    // Main method
    public static void main(String[] args) {
        System.out.println("Hello, KOMA-Script World!");
    }
}
\end{lstlisting}
\lipsum[12]

\addchap*{An Unnumbered Chapter for Special Content}
This is an unnumbered chapter created with \verb|\addchap*|. It won't appear in the ToC by default.
\lipsum[13-14]

\dictum[Albert Einstein]{Imagination is more important than knowledge.}

\appendix
\chapter{Glossary of Terms}
\lipsum[15-16]
\begin{description}
  \item[Typearea] The main text block of a page.
  \item[Recto] A right-hand page.
  \item[Verso] A left-hand page.
\end{description}

\backmatter % Page numbering continues, chapter headings often styled differently (e.g., no number)

\begin{thebibliography}{9}
  \bibitem{koma-guide} Kohm, Markus, and Jens-Uwe Morawski. \textit{The KOMA-Script Guide}. Current version.
  \bibitem{l2h} Lamport, Leslie. \textit{LaTeX: A Document Preparation System}. 2nd ed., Addison-Wesley, 1994.
\end{thebibliography}

\printindex

\end{document}
